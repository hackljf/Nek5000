\section{Two-point flux functions}
Gaussian quadrature on $N$ GLL points exactly integrates a polynomial of order
${2(\!N\!-\!1\!)-1}$. However, nonlinear flux functions like
\ref{convh2to4} produce integrands that are rational functions of $\mathbf{U}$
since we are dividing by density to get velocity from momentum. Gaussian
quadrature at any order is not exact for rational integrands. %PROVE
Worse arithmetic than mere division is required by most state equations. 
integrated on only $N$ points. These errors tend to accumulate with time,
and something must be done to stabilize the scheme against their deleterious
effects.

The major motivation for \S\ref{splitform} is a robust stabilization scheme
for doing quadrature on these fluxes. Equation~\ref{finally2pt} has been
proven to guarantee that if the two-point form $\bF^{\#}$ conserves kinetic energy,
then a high-order SBP operator using it in Equations~\ref{splitr}
through~\ref{splitt} will do so exactly and discretely. %99% sure travis said this p.530
% of JCP 252
More importantly, if $\bF^{\#}$ is \textbf{entropy-stable} (provably increases
physical entropy or conserves it), then high-order SBP operators using it in
DGSEM will be discretely entropy-stable as well. A growing body of tests
\cite{WALCHCOMPARISONTODEAL} has shown that these discrete stability properties
are recovered without loss of formal order of accuracy in constructed solutions
and turbulent flows. Furthermore, these properties translate to better stability
at lower cost than in traditional
``overintegration\cite{KirbyKard}'' or ``dealiasing'' techniques.

We finally give some examples of the two-point flux functions that DGSEM depends on.
As usual, details and motivation may be found in Gassner, Winters and Kopriva\cite{}.
\subsection{Kennedy and Gruber}
The flux in Equation~(3.10) of Gassner, Winters \& Kopriva\cite{} is 
the kinetic-energy-preserving skew-symmetric split form of Kennedy \& Gruber\cite{kg}
reformulated for SBP operators in the form of Equation~\ref{finally2pt}. In
each coordinate direction this flux is, for all 5 conserved variables,
\begin{equation}
\bF_1^{\#}\left(\bu_{(ijk)},\bu_{(ljk)}\right),
\bF_2^{\#}\left(\bu_{(ijk)},\bu_{(ilk)}\right),
\bF_3^{\#}\left(\bu_{(ijk)},\bu_{(ijl)}\right),
\label{someargs}
\end{equation}
\begin{equation}
\bF_1^{\#}=\left[\begin{array}{c} \hat{\rho}\hat{u} \\
                               \hat{\rho}\hat{u}^2 + \hat{p} \\ 
                               \hat{\rho}\hat{u}\hat{v} \\ 
                               \hat{\rho}\hat{u}\hat{w} \\ 
                               \hat{u}\left(\hat{\rho}\hat{e} +\hat{p}\right)
              \end{array}\right], 
\bF_2^{\#}=\left[\begin{array}{c} \hat{\rho}\hat{v} \\
                               \hat{\rho}\hat{v}\hat{u} \\ 
                               \hat{\rho}\hat{v}^2 + \hat{p} \\ 
                               \hat{\rho}\hat{v}\hat{w} \\ 
                               \hat{v}\left(\hat{\rho}\hat{e} +\hat{p}\right)
              \end{array}\right], 
\bF_3^{\#}=\left[\begin{array}{c} \hat{\rho}\hat{w} \\
                               \hat{\rho}\hat{w}\hat{u} \\ 
                               \hat{\rho}\hat{w}\hat{v} \\ 
                               \hat{\rho}\hat{w}^2 + \hat{p} \\ 
                               \hat{w}\left(\hat{\rho}\hat{e} +\hat{p}\right)
              \end{array}\right], 
\label{kennedygruber}
\end{equation}
where the ``hat'' variables are actually function of the indicated quantity
at the two points upon which $\bF^{\#}_k$ acts. For the Kennedy-Gruber and other
energy-stable fluxes,
\begin{equation}
\hat{f}\left(\mathbf{f}_{(ijk)},\mathbf{f}_{(ljk)}\right)\equiv \left\{\left\{
f\right\}\right\}_{((i,l)jk)},
\label{ezhat}
\end{equation}
where Equation~\ref{avgr} defines the ``$\{\{\}\}$'' averaging operator.
Similar definitions exist for the grid lines in other directions on the reference
element. Thus, the Kennedy and Gruber flux is a \textbf{product of averages}
of quantities at the two points used to evaluate $\bF^{\#}$ in Equation~\ref{finally2pt}.
