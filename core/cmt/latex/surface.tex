\section{Surface integral terms}%\label{sfcmath}}\index{Surface\_math@{Surface\_math}}
%INDEX YOUR SHIT!!!
\begin{equation}
I_{\mbox{sfc}}=-\left(\sum_{f=1}^6
\mathbf{E}^{(f)\Txp}\mathbf{B}^{(f)}\left[\mbox{diag}\left(\bhn^{(f)}_j\right)\left(\bh_j-\bh^{\ast(f)}_j\right)\right]\right).
\label{isfc} % was dgweak
\end{equation}
where the ``face'' mass matrix $\mathbf{B}^{(f)}$ is
\begin{equation}
\mathbf{B}^{(f)}\equiv \left[\begin{array}{ccc} \diagdown & & 0 \\ &  \omega_i\omega_j J^{(f)}(\mathbf{x}(r_i,s_j)) &  \\  0 & & \diagdown \end{array}\right]\in\mathbb{R}^{N^2 \times N^2},
\label{massA}
\end{equation}
and we note that the vectors of nodal values $\bhn^{(f)}_j$ and $\bh^{(f)}_j$ in
\ref{isfc} have only $N^2$ elements since they correspond to GLL nodes on face $f$.
Crucially, we have also introduced the \emph{restriction operator} $\mathbf{E}^{(f)}\in\mathbb{R}^{N^2\times N^3}$ for the $f^{\mbox{th}}$ face in \ref{isfc}.
The restriction operator is defined as an indicator that is zero for all GLL nodes except those on the
element faces $\dO_e$, an operation easily expressed as Kronecker products (for example, in the
$r_1-$direction for $f=1$ at $r=r_1=-1$)
\begin{equation}
\bE^{(1)}=\bI\otimes \bI\otimes \mathcal{E}_1
\label{bigE1}
\end{equation}
of the $\bI\in\mathbb{R}^{N\times N}$ identity matrix with unit vectors $\mathcal{E}\in\mathbb{R}^N$
(in \ref{bigE1}, $\mathcal{E}_1\equiv\left(1,0,\dots, 0\right)^{\Txp}$).
Similar indicators may be built up in the $r_2$- and $r_3$-directions by following the ordering of
Kronecker products in \ref{diffkron}.
%Substituting \ref{isfc} into~\ref{sfcintcube} and applying the matrix identity $(\bA\bB)^{\Txp}=\bB^{\Txp}\bA^{\Txp}$ discretely represents the surface integral of the numerical flux on $\Oh$:
Applying the matrix identity $(\bA\bB)^{\Txp}=\bB^{\Txp}\bA^{\Txp}$ to Equation~\ref{isfc} discretely represents the surface integral of the numerical flux on $\Oh$:
\begin{equation}
\sum_{f=1}^6\iint\displaylimits_{-1}^1 v(\br) H^{\ast(f)}_j\hat{n}^{(f)}_j J^{(f)}\, dr \, ds=\sum_{f=1}^6
\bv^{\Txp}\mathbf{E}^{(f)\Txp}\mathbf{B}^{(f)}\left(\left[\mbox{diag}\left(\bhn^{(f)}_j\right)\bh^{\ast(f)}_j\right]\right).
\label{sfcintmatvect}
\end{equation}

\subsection{Surface flux functions}%\label{sfcmath}}\index{Surface\_math@{Surface\_math}}
%INDEX YOUR SHIT!!!
According to a vast body of literature not cited here, summation-by-parts (SBP)
operators need appropriate boundary treatment to behave well. These are
called \textbf{simultaneous approximation terms} (SAT), and the family of
methods that have nice conservation and stability properties always have these
two things together and are called ``SBP-SAT'' methods. For the purposes of
DGSEM, Gassner, Winters \& Kopriva \cite[Eq]{} recommend a numerical flux
function consisting of a symmetric function and an extra dissipative term:
\begin{equation}
\bH^{\ast}\left(\bU^-,\bU^+\right)=\mathbf{F}^{\#}\left(\bU^-,\bU^+\right)+\mathbf{F}_{\mbox{stab}}\left(\bU^-,\bU^+\right),
\label{numfluxfinally}
\end{equation}
where $\bH^{\ast}$ is the flux between neighboring elements
in Equation~\ref{dghstrong}. Gassner, Winters and Kopriva have indeed recycled
the same split-form flux function used in Equations~\ref{splitr} through~\ref{splitt},
but instead of evaluating it at two separate points, they evaluate it using the
two values present at each interface between elements.

SAT tend to be dissipative, and the dissipative nature of $\bH^{\ast}$ in
DG is needed in DGSEM as well. $\bF^{\#}$ is augmented with a stabilizing
flux $\bF_{\mbox{stab}}$. Examples of this will be given in \S.
