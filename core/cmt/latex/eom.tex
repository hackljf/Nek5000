\section{Euler equations of gas dynamics\label{sec:gaseom}}
We now present the Euler equations of gas dynamics
In this section, bold-faced quantities are vectors in $\mathbb{R}^3$ except for the conserved variables
$\bU$,
\begin{equation}
  \mathbf{U}=\left[\begin{array}{l} \phi_g \rho \\ \phi_g \rho u \\ \phi_g \rho v \\ \phi_g \rho w \\ \phi_g \rho E \end{array}\right],
  \label{uvect}
\end{equation}
which live in $\mathbb{R}^5$. Conserved variables and their inviscid fluxes\footnote{\textcolor{red}{but not their viscous fluxes}}
are weighted by the \textbf{gas volume fraction} $\phi_g$. To more easily subscript flux vectors we write the $m$\nth component of $\bU$ as being governed by the
conservation law
\begin{equation}
%\pp{U_m}{t} + \divnce \bH_m=0,\, m=1,\dots,5.%R_m, I don't like having a period after an integer
\pp{U_m}{t} + \divnce \bH_m=R_m,\, m\in\left[1,5\right].%R_m,
\label{conslaw}
\end{equation}
The gas velocity $\mathbf{v}$ is % and spatial coordinate $\mathbf{x}$ are
\begin{equation}
\mathbf{v}=\left[\begin{array}{l} u \\ v \\ w \end{array}\right]=
\left[\begin{array}{l} v_1 \\ v_2 \\ v_3 \end{array}\right],%, \qvad
%\mathbf{x}=\left[\begin{array}{l} x \\ y \\ z \end{array}\right]=
%\left[\begin{array}{l} x_1 \\ x_2 \\ x_3 \end{array}\right],
\label{uxpedantic}
\end{equation}
and $\rho$ is the gas density, $E$ is the mass-specific total energy $e+\frac{1}{2}|\mathbf{v}|^2$ of
the gas, $e$ is the gas internal energy, and $p$ is the thermodynamic gas pressure.

$\bH_m(\bU,\nabla \bU) \mathbb{R}^{5\times 3}\rightarrow \mathbb{R}^3$
is the flux vector of equation~$m$. Like \ref{convanddiff},
$\bH_m$ is made up of a convective flux $\bH_{m,c}(\bU)$ and a
diffusive flux $\bH_{m,d}(\bU,\nabla \bU)$. We consider the convective
fluxes first. For gas density $\bH_{1,c}(\bU)$ is
\begin{equation}\label{convh1}
   \bH_{1,c}=\phi_g\rho \mathbf{v}= \left[U_2,U_3,U_4\right]^{\Txp}.
\end{equation}
For gas momentum $U_{2-4}$, the convective fluxes are
\begin{align}
   \bH_{2,c}=\phi_g\left[\begin{array}{l} \left(\rho u\right)u \!+\! p \\
                                               \left(\rho u\right)v \\
                                               \left(\rho u\right)w \end{array}\right],
   \bH_{3,c}=\phi_g\left[\begin{array}{l} \left(\rho v\right)u \\
                                               \left(\rho v\right)v \!+\! p \\
                                               \left(\rho v\right)w \end{array}\right], %\nonumber \\
   \bH_{4,c}=\phi_g\left[\begin{array}{l} \left(\rho w\right)u \\
                                               \left(\rho w\right)v \\
                                               \left(\rho w\right)w \!+\! p \end{array}\right],  \label{convh2to4}
\end{align}
and, for total energy $\rho E$,
\begin{equation} \label{convh5}
   \bH_{5,c}= \phi_g \mathbf{v}\left(\rho E +p\right).
\end{equation}

The system is closed by an equation of state,
\begin{equation}
\left[p,T\right]=\mbox{EOS}\left(\rho,e\right).
\label{eos}
\end{equation}
Internal energy per
unit mass $e=E-\frac{1}{2}|\mathbf{v}|^2$ is related to gas temperature $T$ by the intensive property $c_v$, the constant-volume specific heat, such that
\begin{equation}
   e=\int c_v(T)dT.
   \label{cvt}
\end{equation}
Generally, \ref{cvt} must be solved for temperature $T$ implicitly, iteratively,
or via tabulation.
For calorically perfect gases, $c_v$ is constant. For both thermally and
calorically perfect gases, pressure is obtained last via
\begin{equation}
p=\rho R T,
\label{eos_tpg}
\end{equation}
where the specific gas constant $R=\left(\gamma-1\right)c_v$ requires the specification
of $\gamma=c_p/c_v$, the ratio of constant-pressure specific heat $c_p$ to $c_v$. Finally, the sound
speed is
\begin{equation}
a=\sqrt{\frac{\gamma p}{\rho}}.%=\sqrt{\gamma RT}.
\label{asndtpg}
\end{equation}
